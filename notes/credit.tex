\section{Project: Credit risk for mortgages}

Consider a bank that must design a decision rule for giving loans to individuals. In this particular case, some of each individual's characteristics are partially known to the bank.  We can assume that the insurer has a linear utility for money and wishes to maximise expected utility. Assume that the $t$-th individual is associated with relevant information $x_t$, sensitive information $z_t$ and a potential outcome $y_t$, which is whether or not they will default on their mortgage. For each individual $t$, the decision rule chooses $a \in \CA$ with probability $\pol(a_t = a \mid x_t)$.

As an example, take a look at the historical data in \texttt{data/credit/german.data-mumeric}, described in \texttt{data/credit/german.doc}. Here there are some attributes related to financial situation, as well as some attributes related to personal information such as gender and marital status. 

A skeleton for the project is available at \url{https://github.com/olethrosdc/ml-society-science/tree/master/src/project-1}. Start with \verb|random_banker.py| as a template, and create a new module \verb|name_banker.py|. You can test your implementation with the \verb|TestLending.py| program. 

For ensuring progress, the project is split into two parts:
\subsection{Deadline 1: September 14}
The first part of the project focuses on a baseline implementation of a banker module.
\begin{enumerate}
\item Design a policy for giving or denying credit to individuals, given their probability for being credit-worthy. Assuming that if an individual is credit-worthy, you will obtain a return on investement of $r = 0.5\%$ per month. Take into account the length of the loan to calculate the utility through \verb|NameBanker.expected_utility()|. Assume that the loan is either fully repaid at the end of the lending period $n$, or not at all to make things simple. If an individual is not credit-worthy you will lose your investment of $m$ credits, otherwise you will gain $m [(1 + r)^{n} - 1]$ . Ignore macroenomic aspects, such as inflation. In this section, simply assume you have a model for predicting creditworthiness as input to your policy, which you can access \verb|NameBanker.get_proba()|. 
\item Implement \verb|NameBanker.fit()| to fit a model for calculating the probability of credit-worthiness from the german data. Then implement \verb|NameBanker.predict_proba()| to predict the probability of the loan being returned for new data. What are the implicit assumptions about the labelling process in the original data, i.e. what do the labels represent?
\item Combine the model with the first policy to obtain a policy for giving credit, given only the information about the individual and previous data seen. In other words, implement \verb|Namebanker.get_best_action()|.
\item Finally, using \verb|TestLending.py| as a baseline, create a jupyter notebook where you document your model development. Then compare your model against \verb|RandomBanker|.
\end{enumerate}

\subsection{Deadline 2: September 28}
The second part of the project focuses on issues of reproducibility, reliability, privacy and fairness. That is, how desirable would it be to use this model in practice? Here are some sample questions that you can explore, but you should be free to think about other questions.
\begin{enumerate}
\item Is it possible to ensure that your policy maximises revenue? How can you take into account the uncertainty due to the limited and/or biased data? What if you have to decide for credit for thousands of individuals and your model is wrong? How should you take that type of risk into account?\footnote{You do not need to implement anything specific for this to pass the assignment, but you should outline an algorithm in a precise enough manner that it can be implemented. In either case you should explain how your solution mitigates this type of risk.}
\item Does the existence of this database raise any privacy concerns? If the database was secret (and only known by the bank), but the credit decisions were public, how would that affect privacy? (a) Explain how you would protect the data of the people in the training set. (b) Explain how would protect the data of the people that apply for new loans. (c) \emph{Implement} a private decision making mechanism for (b),\footnote{If you have already implemented (a) as part of the tutorial, feel free to include the results in your report.} and estimate the amount of loss in utility as you change the privacy guarantee.
\end{enumerate}

\subsection{Deadline 3: October 10}

 Choose one concept of fairness, e.g. balance of decisions with respect to gender. How can you measure whether your policy is fair? How does the original training data affect the fairness of your policy? To help you in this part of the project, here is a list of guiding questions.

 \begin{itemize}
 \item Identify sensitive variables. Do the original features already imply some bias in data collection?
 \item Analyse the data or your decision function with simple statistics such as histograms.
 \item For balance (or calibration), measure the total variation of the action (or outcome) distribution for different outcomes (or actions) when the sensitive variable varies.
 \item Advanced: What would happen if you were looking at fairness by also taking into account the amount of loan requested?
 \item Advanced: Using stochastic gradient descent, find a policy that balances out fairness and utility.
 \end{itemize}

Submit a final report about your project, either as a standalone PDF or as a jupyter notebook. For this, you can imagine playing the role of an analyst who submits a possible decision rule to the bank, or the authorities. You'd like to show that your decision rule is quite likely to make a profit, that it satisfies \emph{some} standard of privacy and that it does not unduly discriminate between applicants. You should definitely point out any possible \emph{deficiencies} in your analysis due to your assumptions, methodology, or the available data from which you are drawing conclusions.

%%% Local Variables:
%%% mode: latex
%%% TeX-master: notes
%%% End:
